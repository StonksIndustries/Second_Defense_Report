% This file CAN be compiled on its own
% Its output is included into the Google Sheet


\documentclass[border=2 3]{standalone}

\usepackage{pgfgantt}

\usepackage{geometry}

\geometry{margin=2.5cm}

\begin{document}
\centering
  \begin{ganttchart}[
    hgrid,
    vgrid,
    % Today paramaters
    % today=17,
    % today rule/.style={blue},
    group left shift=0,
    group right shift=0,
    group peaks tip position=0,
    group height=.3,
    x unit=0.6cm,
    y unit chart=0.7cm,
    bar/.append style={fill=green!50}, 
    ]{1}{18}
    \gantttitlelist{"Oct.", "Nov.", "Dec.", "Janv.", "Fevr", "Mars", "Avril", "Mai", "Juin"}{2} \\
    
    % \ganttgroup{Concept}{1}{4} \\
    \ganttbar[progress=100]{Cahier des charges}{1}{2} \\
    \ganttbar[progress=100]{Recherche du concept}{1}{2} \\
    \ganttbar[progress=100]{\'Ecriture du scnénario}{3}{4} \\  
    \ganttnewline
    
    % \ganttgroup{Design}{5}{10} \\
    \ganttbar[progress=100]{Design des textures}{4}{12} \\
    \ganttbar[progress=100]{Integration dans godot}{9}{15} \\
    \ganttbar[progress=100]{Validation}{11}{13} \\
    \ganttnewline
    
    % \ganttgroup{Programmation du jeu}{9}{14} \\
    \ganttbar[progress=100]{Premiers tests sur Godot}{5}{7} \\
    \ganttbar[progress=100]{Développement}{6}{17} \\
    \ganttbar[progress=100]{Réseau}{6}{12} \\
    \ganttbar[progress=100]{IA}{12}{13} \\
    \ganttbar[progress=100]{Phase de débug / corection}{16}{17} \\
    \ganttnewline
    
    % \ganttgroup{Site web}{5}{8} \\
    \ganttbar[progress=100]{Design du site web}{6}{7} \\
    \ganttbar[progress=100]{Développement}{7}{14} \\
    \ganttbar[progress=100]{Phase de débug / corection}{15}{15} \\
    \ganttnewline
    \ganttvrule[
      % Soutenance orale : 22 janvier 2024
      vrule/.append style={red, thin},
      vrule offset=.2,01
      vrule label node/.append style={anchor=north west}
      ]{Oral}{8}
      \ganttvrule[
        % Soutenance 1 : 18 mars 2024
        vrule/.append style={red, thin},
      vrule offset=.2,01
      vrule label node/.append style={anchor=north west}
      ]{Soutenance 1}{12}
      \ganttvrule[
        % Soutenance 2 : 17 juin 2024
        vrule/.append style={red, thin},
        vrule offset=.2,01
        vrule label node/.append style={anchor=north west}
        ]{Soutenance 2}{18}
        
      \end{ganttchart}

    
    \end{document}