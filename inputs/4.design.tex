% Written by Louise Fussien

\subsection{Inspirations Design} % TODO: Rename

En plus d'évoquer la nostalgie des jeux classiques, le pixel art nous offre une exigibilité artistique tout en étant relativement léger en termes de ressources.
Inspiré par des jeux emblématiques tels que Stardew Valley, Undertale, ainsi que par d'autres jeux rétro, le choix du style graphique pixel art en 2D pour Lands of Azerith provient du choix réfléchi de l'esthétique du jeu et son potentiel à immerger les joueurs dans un monde magique et riche en aventures.
En effet, ce choix s'est révélé être une décision artistique judicieuse.
Ce style graphique simple et détaillé à la fois confère au jeu un charme unique tout en permettant une représentation qualitative des environnements et des personnages.
De plus, l'atmosphère visuelle du jeu est soigneusement choisie en fonction des différentes zones du jeu pour une meilleure immersion des joueurs dans l'univers apporté par Lands of Azerith.
\\

\subsection{Création et implémentation design} % TODO: Rename

Le choix du style graphique pixel art en 2D pour Lands of Azerith provient du choix réfléchi de l'esthétique du jeu et son potentiel à immerger les joueurs dans un monde magique et riche en aventure.
En effet, ce choix s'est révélé être une décision artistique judicieuse.
Ce style graphique simple et détaillé à la fois confère au jeu un charme unique tout en permettant une représentation qualitative des environnements et des personnages.
De plus, l'atmosphère visuelle du jeu est soigneusement choisie en fonction des différentes zones du jeu pour une meilleure immersion des joueurs dans l'univers apportée par Lands of Azerith.
\\

\subsection{Création des textures}

Création des textures La création des textures pour Lands of Azerith a été une entreprise passionnante et méticuleuse, réalisée principalement à l'aide d'outils de retouche d'image comme GIMP.
Cette approche a permis à notre équipe de bénéficier d'une grande souplesse créative tout en exploitant les fonctionnalités pratiques de ce logiciel open-source.
Le processus de création des textures commence par la conceptualisation, où nous explorons différentes idées et concepts pour chaque élément du jeu, qu'il s'agisse d'environnements, de personnages ou d'objets.
Une fois le concept finalisé et validé, nous passons à l'étape de la concrétisation, où le directeur artistique réalisera les textures.
Pour les réaliser, nous nous inspirons d'image ainsi que de texture d'autre jeu existant correspondant au style choisi, faisant ainsi office de modèle adapté à notre projet.
L'intégration de détails et de textures nécessite souvent plusieurs itérations et ajustements pour parfaire le rendu final.
Nous accordons également de l'attention aux petits détails qui contribuent à enrichir l'expérience visuelle du joueur, qu'il s'agisse des textures de sol, la fluidité des animation des entités, ou encore la création de variantes de certaines textures an de casser la monotonie du décor.
\\

Chaque texture est ensuite implémentée et testée dans le jeu pour évaluer son apparence et son intégration dans l'environnement de jeu.
Des ajustements supplémentaires sont apportés si nécessaire pour garantir une cohérence visuelle et une immersion optimale.
En fin de compte, notre objectif est de créer des textures qui non seulement embellissent le monde de Lands of Azerith, mais qui racontent également une histoire et enrichissent l'expérience de jeu.
Grâce à notre engagement ainsi qu'à notre rigueur dans le processus de création, nous sommes convaincus que les textures que nous avons créées contribuent de manière significative au succès et à l'attrait de notre jeu.
Intégration des éléments graphiques dans le jeu L'intégration des textures et des éléments graphiques dans Lands of Azerith est une étape cruciale pour créer une expérience immersive et cohérente pour les joueurs.
\\

Notre processus d'intégration vise à garantir que chaque texture s'intègre harmonieusement dans l'environnement de jeu, renforçant ainsi l'atmosphère et l'esthétique globale du monde virtuel.
Lors de l'intégration des textures, nous nous assurons d'abord de respecter les spécifications techniques du jeu, en veillant à ce que les dimensions, les formats et les résolutions des textures correspondent aux exigences des développeurs.
Une fois cette étape technique achevée, nous procédons à l'incorporation des textures dans les différents éléments du jeu, tels que les décors, les personnages et les objets, afin de donner vie à la carte dans laquelle le joueur va explorer.
La conception de la carte de Lands of Azerith a été un processus méticuleux et essentiel, visant à créer un monde cohérent et immersif pour les joueurs à explorer.
Tout d'abord, une planification initiale a été effectuée sur papier pour définir les principaux éléments de la carte, tels que les zones géographiques, les villes et les donjons.
Cette étape a permis de définir la structure globale du monde du jeu.
Ensuite, les différentes zones géographiques ont été créées avec soin, en tenant compte de leur positionnement relatif et de leur connexion les unes aux autres.
Chaque zone, qu'il s'agisse de forêts, de montagnes, de plaines ou de marais, a été imaginée avec ses propres caractéristiques visuelles et thématiques pour offrir une variété d'environnements aux joueurs.
\\

Des points d'intérêt ont été ensuite dispersés à travers la carte pour encourager l'exploration et offrir des récompenses aux joueurs.
Cela inclut des villages, des temples, des grottes, des trésors cachés et d'autres éléments interactifs qui enrichissent l'expérience de jeu.
Par la suite, les routes et les chemins ont été définis pour relier les différentes zones entre elles, offrant aux joueurs des voies de passage pour naviguer dans le monde du jeu.
Des itinéraires principaux aux sentiers secrets, chaque chemin a été conçu pour offrir une expérience de voyage variée et engageante.
Et c'est en suivant le plan détaillé de la carte que celle-ci est implémentée.
Enfin, une phase d'équilibrage et d'ajustements a été réalisée, où des tests ont été effectués pour évaluer l'équilibre du monde du jeu en termes de difficulté, de progression et d'accessibilité.
Des ajustements ont été apportés pour garantir une expérience équilibrée et satisfaisante pour les joueurs.
En intégrant de manière réfléchie les textures et les éléments graphiques dans Lands of Azerith, nous sommes contents de créer un monde visuellement captivant et immersif qui ravira les joueurs et les plongera dans une aventure épique.

\subsection{Musique et bruitage}

\subsubsection{Musiques}

Le choix de l'électro, plus précisément de la Dance électro, pour la bande sonore de Lands of Azerith provient de son dynamisme et de sa capacité à renforcer l'énergie et l'immersion du jeu. 
En sélectionnant des compositions d'artistes comme Cosmograph et Zekk, nous avons assuré une harmonie entre l'audio et les visuels du jeu, offrant ainsi une expérience sensorielle complète et captivante.

\subsubsection{Bande sonore}

La bande sonore de Lands of Azerith est composée de 19 musiques existantes, sélectionnées spécifiquement pour enrichir les biomes et les situations du jeu. 
Chaque biome dispose de trois versions de musique : jour, nuit et combat, adaptées pour capturer l'ambiance et l'intensité des différentes phases du gameplay.

\subsubsection{Exemples de Musiques de Biomes}

\begin{itemize}
    \item La musique d'ambiance de jour dans les plaines d'Emberwood capture l'atmosphère sereine et pastorale du paysage, offrant aux joueurs une immersion tranquille dans cet environnement ouvert.
    \\

    \item La version nocturne de la même musique intensifie le ton, créant une atmosphère mystérieuse et potentiellement dangereuse alors que les menaces nocturnes émergent 
    \\

    \item Pendant les combats, la musique devient plus dynamique et rythmée pour stimuler l'adrénaline des joueurs, les incitant à réagir avec précision et stratégie

\end{itemize}


\subsubsection{Musiques des Boss et Zones Pacifiques}

Chaque boss est accompagné d'une musique distinctive et immersive, conçue pour refléter la menace et l'importance narrative de ces rencontres épiques. 
Ces compositions musicales renforcent l'intensité des combats tout en enrichissant l'expérience émotionnelle des joueurs à travers des thèmes mémorables.

\subsubsection{Bruitages en Jeu}

Les bruitages en jeu, tels que les coups d'épée, les pas et les sons d'inventaire, sont minutieusement choisis pour enrichir l'immersion des joueurs. 
Chaque son contribue à l'ambiance visuelle et narrative du jeu, garantissant une expérience immersive et engageante à travers des détails sonores qui complètent parfaitement le monde visuel de Lands of Azerith.

\subsubsection{Voix Off}

La voix off était initialement prévue pour offrir une narration riche et immersive tout au long de Lands of Azerith. Cependant, en raison de contraintes techniques LIES AU DEPART DE LA PERSONNE CENSÉ REALISER LES VOIX cette fonctionnalité a été modifiée. La narration et les éléments narratifs sont maintenant intégrés dans l'environnement visuel et musical du jeu, renforçant l'immersion des joueurs à travers des visuels évocateurs et des thèmes musicaux dynamiques.
