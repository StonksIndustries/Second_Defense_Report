% Written by Louise Fussien

% This file CAN NOT be compiled on its own
% It is included by ../Book_of_Specifications.tex

\subsection{Objectif de la soutenance et du projet}



Dans le cadre de notre formation, nous avons entrepris un projet ambitieux visant à développer un jeu vidéo.
Ce projet, qui s'étend sur l'ensemble de l'année académique, nous a permis d'acquérir et de perfectionner nos compétences en programmation,
en design de jeux et en gestion de projet. La soutenance de fin d'année marque l'aboutissement de nos efforts collectifs et individuels.
Elle offre une occasion privilégiée de présenter l'évolution du projet, les défis surmontés et les résultats obtenus.
\\

Nous sommes fiers de notre travail et des obstacles que nous avons surmontés pour réaliser \textit{Lands of Azerith},
et nous sommes impatients de partager notre expérience avec vous.
\\

Ce rapport de soutenance, élaboré par l'équipe composée de Louise Fussien, Martin Pasquier, Ayemane Bouarbi et Michaël Museux, présente
en détail le projet \textit{Lands of Azerith}, un jeu de plateforme multijoueur innovant développé dans le cadre de notre formation. \textit{Lands of Azerith}  se distingue par son concept unique
où les joueurs évoluent dans un monde en 2D, surmontant des obstacles et résolvant des énigmes pour progresser dans le jeu.
\\

Le développement de \textit{Lands of Azerith}  s'est déroulé en plusieurs phases, marquées par des défis techniques et des moments de réflexion collective.
La première étape a consisté en une analyse approfondie des mécaniques de jeu existantes, suivie d'une étude de faisabilité pour définir les grandes lignes du projet.
Ensuite, nous avons élaboré un cahier des charges détaillé, précisant les fonctionnalités principales du jeu, les contraintes techniques, et les délais à respecter.
\\

Chaque membre de l'équipe a apporté ses compétences spécifiques, qu'il s'agisse de programmation, de design graphique, de création sonore, ou de développement de
l'intelligence artificielle. Cette collaboration étroite nous a permis de surmonter de nombreux obstacles et de progresser de manière constante vers notre objectif.
Nous avons utilisé des outils tels que Godot pour le développement du jeu, et GDU pour la modélisation en 2D, guidés par la volonté de créer un produit final de haute qualité.
\\

Ce rapport de soutenance décrit en détail chaque aspect du projet
\textit{Lands of Azerith}, depuis sa conception initiale jusqu'à sa réalisation finale.
Il inclut des sections sur les choix technologiques, la répartition des tâches, les défis rencontrés, et les solutions apportées. En documentant ce processus,
nous espérons fournir une vue d'ensemble claire et complète de notre travail, ainsi qu'un témoignage de notre capacité à travailler en équipe et à innover dans le domaine du
jeu vidéo.






\subsection{Synopsis du jeu}

\textit{Lands of Azerith} est un jeu d'aventure immersif où le joueur incarne un héros cherchant à libérer les terres d'\textit{Azerith} de l'emprise tyrannique d'un empire dictatorial.
L'histoire principale se déroule en trois actes, mais pour l'instant, nous nous concentrerons uniquement sur l'acte 1, qui est le seul développé et forme la base du jeu.
Voici une description détaillée de l'acte 1, avec tous ses éléments, mécaniques et quêtes :

\subsubsection*{Acte 1 : Introduction et Tutoriel (Emberwood)}

L'acte 1 commence dans le village pittoresque d'Emberwood, une petite communauté nichée au milieu des ruines celtiques.
Ce village sert de point de départ pour le joueur et est conçu comme une zone de tutoriel où le joueur peut se familiariser avec les différentes mécaniques du jeu.
Voici les éléments clés de cette première partie :
\\

\begin{itemize}
    \item Choix de la Classe de Personnage : Le joueur peut choisir parmi 9 classes différentes, chacune avec ses propres compétences et styles de jeu. Les classes comprennent des guerriers, des mages, des archers, et d'autres archétypes classiques, offrant une variété de choix stratégiques.
          \\
\end{itemize}

\textbf{Apprentissage des Mécaniques du Jeu :}
\\

\begin{itemize}
    \item Maniement des Armes : Le joueur apprend à manier différentes armes comme les haches, épées, et arcs.
          Chaque arme a ses propres avantages et inconvénients, et le choix de l'arme dépend de la classe choisie.
          \\

    \item Équipements par Classe : Le joueur découvre les différents types d'armures (lourdes, légères) adaptées à chaque classe, influençant la mobilité et la protection.
          \\

    \item Éléments : Le jeu introduit les éléments (feu, eau, vent), chacun ayant des effets spécifiques en combat et sur l'environnement.
          \\

    \item Consommables : Le joueur apprend à utiliser des potions de régénération, de défense, et d'autres consommables pour survivre et se renforcer.
          \\

    \item Commerce en Ville : Interactions avec les forgerons, marchands, et autres commerçants pour acheter et améliorer l'équipement.
          \\

    \item Mouvements du Personnage : Le joueur apprend à se déplacer, sauter, courir, et interagir avec l'environnement.
          \\
\end{itemize}

\textbf{Combat Tutoriel :}

\begin{itemize}
    \item À la fin du tutoriel, le joueur participe à un combat d'entraînement où il met en pratique la gestion des mouvements, l'utilisation des compétences, des pouvoirs élémentaires, des armes, et des consommables.
          Ce combat sert de test final pour les compétences acquises.
          \\
\end{itemize}

\textbf{Lore et Contexte :}

\begin{itemize}
    \item En plus des mécanismes de jeu, cette première partie introduit les bases du lore de \textit{Lands of Azerith}, plaçant le joueur dans un monde riche en histoire et en mystères à découvrir.
\end{itemize}

\subsection{Notre entreprise}

La prestigieuse Stonks Industries, fondée en 2023 par un groupe de cinq étudiants, s'est érigée en un pilier de l'industrie du jeu vidéo en l'espace d'un mois seulement.
Animée par une passion débordante pour l'innovation et la créativité, l'entreprise a su imposer sa vision novatrice.
Guidée par l'audace de ses fondateurs, elle s'est donnée pour noble dessein de réinventer l'expérience ludique à travers des récits immersifs, des visuels à couper le souffle et un gameplay captivant, incarné notamment par leur projet phare, \textit{Lands of Azerith}.
\\

Issue d'une humble salle de l'EPITA, les origines modestes de la Stonks Industries témoignent de sa volonté farouche de repousser les frontières de l'imaginaire et de l'innovation.
Implantée en plein cœur de la capitale, Paris, l'entreprise est saluée pour son engagement à repousser les limites de la technologie et du gameplay afin de créer des expériences immersives et captivantes qui transcendent les générations.
À sa tête, Martin Pasquier, en qualité de directeur général, insuffle à l'entreprise une vision visionnaire et un leadership éclairé.

\subsection{L'équipe}

Avant de poursuivre, permettez-nous de vous présenter les personnalités fascinantes qui insufflent vie et passion à la Stonks Industries.
Chacun d'eux apporte une expertise unique et une vision singulière au sein de notre équipe.
Leurs contributions exceptionnelles ont façonné l'identité de notre entreprise et ont contribué à notre succès fulgurant dans l'industrie du jeu vidéo.
Découvrez les portraits et les parcours inspirants de ces figures emblématiques qui font de la Stonks Industries un foyer d'innovation et de créativité.

\subsubsection{Ayemane Bouarbi}
En tant que lead game artist, Ayemane, étudiant de génie à EPITA, à 18 ans seulement, se voit confier la lourde responsabilité de piloter la direction artistique du projet de jeu vidéo    ambitieux \textit{Lands of Azerith}.
Et ce choix n'a pas été realisé sans raison : Ayemane est un artiste du jeu vidéo ayant contruit sa propre réputation avec ses plus de 10 années d'expérience dans l'industrie.
En effet, passionné par les jeux vidéo et de dessin depuis l'enfance, il commence à créer des jeux vidéo sur scratch dès l'âge de 7 ans avant de commencer à en faire sur python, et plus tard, en C++.
Il a également de l'expérience dans la création de concepts art, la modélisation 3D, le texturing, le shading et le lighting.
Dans son parcours professionnel, il a participé à la direction artistique de plusieurs jeux vidéos, que cela soit avec de petits studios indépendants ou bien comme sur certains jeux classés AAA.
Dans son rôle actuel de Lead Game Artist chez les Studios Stonks Industries, il est responsable de la gestion et de la vision artistique du jeu, ainsi que de la supervision de la création des assets et des textures du jeu.

\subsubsection{Louise Fussien}

Louise est fascinée par le numérique qui pour elle, est la fusion entre la science et l'art, entre la logique et l'intuition.
C'est un domaine qui requiert à la fois des compétences techniques pointues, mais aussi une grande créativité.
Louise s'est donc tout naturellement tournée vers des spécialités scientifiques au lycée et en particulier a eu l'occasion de concevoir des jeux vidéo et de s'initier à la programmation avant d'intégrer EPITA.
En dehors de son travail, Louise pratique le karaté depuis l'âge de 8 ans en club et lors de nombreuses compétitions de combat.
Cet art martial lui a appris l'exigence, la gestion de la pression et le travail en équipe, qualités essentielles pour travailler au sein de la Stonks Industries.
Louise parle, en plus de l'anglais, l'espagnol et le russe. Elle n'en délaisse pas moins sa langue maternelle,
le français et a un fort attrait pour la littérature française dont la richesse et la beauté des textes ont développé une part de sa créativité.

\subsubsection{Michaël Museux}

Michaël Museux, âgé de 17 ans, est étudiant à l'EPITA en première année de classe préparatoire intégrée.
Depuis un très jeune âge, il est passionné par l'informatique, et plus particulièrement par la programmation.
Après avoir découvert Scratch, il a par exemple écrit des programmes pour résoudre des calculs complexes en mathématique, parfois même impossibles à faire à la main. Michaël est aussi passionné par l'algorithmie, son excellent niveau dans ce domaine lui a même permis de participer à des concours d'algorithmie, comme le concours Castor ou Algorea où il s'est classé parmi les meilleurs, alors qu'il n'était qu'au lycée.
Attiré par l'idée de se faire ses propres outils, il s'est aussi intéressé à l'intelligence artifcielle (IA) et notamment les technologies d'automatisation.
En effet, à une époque où la popularité des IAs grandissait de plus en plus, il semblait évident pour lui de s'y intéresser, et peut-être même d'en faire son métier.
Par évidence, Michaël intègre l'équipe de développement de Stonks Industries, qui commence le développement d'un nouveau jeu : \textit{Lands of Azerith}. 
Son rôle dans l'équipe est donc tout trouvé, il sera le développeur principal du jeu, avec un rôle important dans l'intelligence artificielle du jeu. 
De plus, ses quelques compétences en développement web lui permettent de participer au développement du site web de l'entreprise.

\subsubsection{Martin Pasquier}

Originaire de Normandie, jeune provincial déraciné, Martin Pasquier est le directeur technique du projet, à l'âge de 17 ans, il a déjà une bonne expérience dans le domaine de la programmation et du développement web.
En tant que directeur technique, il est responsable de la gestion de l'équipe de développement et de la planification des tâches. Il est également responsable de la conception et de la mise en oeuvre du site web.
Sa passion pour la programmation ayant commencé à un age où le jeu vidéo était très présent dans sa vie. Il semble évident que Martin a toujours voulu concevoir son propre jeu vidéo.
De par son attrait pour la culture médiévale fantastique, il assistera sans problème le game designer en prenant part à la conception du scenario et de l'histoire du jeu.
Martin est également un grand fan de musique, il aime écouter tous types de musique, ce qui lui sera utile pour la partie sonore du jeu.
La diversité de ses goûts musicaux lui permettra de créer une bande sonore riche et variée, qui s'adaptera parfaitement à l'ambiance du jeu.

\subsection{Changements depuis la dernière soutenance}


Depuis la dernière soutenance, plusieurs changements significatifs ont eu lieu au sein de notre équipe et dans le développement du projet.
Tout d'abord, nous avons dû faire face au départ d'Alexandre Colsch, qui a quitté EPITA  pour des raisons personnelles.
Ce départ a été compensé par l'arrivée de Louise Fussien, dont les compétences en programmation et design ont été précieuses pour la progression du projet.
\\

Ce changement fait suite à une autre transition importante : Mohamed Aziz ben Amor, un membre clé de l'équipe, était parti juste avant notre première soutenance.
La dynamique de notre équipe a donc été profondément impactée, nécessitant une adaptation rapide et une redistribution des responsabilités.
\\

Malgré ces bouleversements, nous avons réussi à maintenir un rythme de travail soutenu et à progresser de manière significative.
\\

\subsection{Annonce de plan}

Dans ce cahier des charges, nous allons vous présenter un compte rendu détaillé de l'avancement de notre projet, en décrivant les différentes étapes de son développement, les obstacles rencontrés et les solutions mises en place.
Nous aborderons également les aspects organisationnels et techniques qui ont structuré notre travail.
\\

Tout d'abord, nous discuterons des deadlines et de l'organisation.
Nous commencerons par une analyse des objectifs initiaux et des progrès réalisés jusqu'à présent.
Ensuite, nous ferons un point sur l'état actuel du projet. Nous expliquerons également les délais que nous avons rencontrés et les raisons pour lesquelles ils ont été nécessaires.
\\

Ensuite, nous examinerons l'organisation du groupe. Nous décrirons les méthodes de communication utilisées au sein de l'équipe pour garantir une collaboration efficace.
Nous présenterons également la répartition des tâches entre les différents membres de l'équipe, et comment nous avons utilisé GitHub pour la gestion du code source et le versionnage.
\\

Nous aborderons ensuite le nouveau cahier technique, où nous fournirons une description détaillée des aspects techniques du projet.
Cela inclura des sections spécifiques sur les différents modules et composants que nous avons développés.
\\

Pour le design, nous partagerons nos sources d'inspiration et expliquerons le processus de création et d'implémentation de l'interface et de l'expérience utilisateur.
Nous traiterons également de l'intégration du son et d'autres éléments multimédias.
\\

Enfin, nous discuterons du site web du projet.
Nous expliquerons l'objectif du site web, son développement et son apparence finale.
Nous aborderons également la question du responsive design et de l'optimisation pour les appareils mobiles.
\\

La conclusion fournira un récapitulatif de tout ce que nous avons accompli, ainsi que des remerciements aux personnes et aux ressources qui nous ont soutenus tout au long de ce projet.
\\

Nous inclurons également des annexes pour fournir des informations supplémentaires et les documents pertinents pour une compréhension complète du projet.
\\

Ce document vise à fournir une vue d'ensemble complète et détaillée de notre travail, en mettant en lumière notre méthodologie, nos défis, et les solutions que nous avons trouvées pour les surmonter.
