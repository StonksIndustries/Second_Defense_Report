% Written by Louise Fussien

% This file CAN NOT be compiled on its own
% It is included by ../Book_of_Specifications.tex

\subsection{Objectif de la soutenance et du projet}

Dans le cadre de notre formation à EPITA, nous avons entrepris un projet ambitieux visant à développer un jeu vidéo. 
Ce projet, qui s'étend sur l'ensemble de l'année académique, nous a permis d'acquérir et de perfectionner nos compétences en programmation, en design de jeux et en gestion de projet. La soutenance de fin d'année marque l'aboutissement de nos efforts collectifs et individuels. 
Elle offre une occasion privilégiée de présenter l'évolution du projet, les défis surmontés et les résultats obtenus. 
\\

\subsection{Synopsis du jeu}

Lands of Azerith est un jeu d'aventure immersif où le joueur incarne un héros cherchant à libérer les terres d'Azerith de l'emprise tyrannique d'un empire dictatorial. 
L'histoire principale se déroule en trois actes, mais pour l'instant, nous nous concentrerons uniquement sur l'acte 1, qui est le seul développé et forme la base du jeu. 
Voici une description détaillée de l'acte 1, avec tous ses éléments, mécaniques et quêtes :

\subsection*{Partie 1 : Introduction et Tutoriel (Emberwood)}

L'acte 1 commence dans le village pittoresque d'Emberwood, une petite communauté nichée au milieu des ruines celtiques. Ce village sert de point de départ pour le joueur et est conçu comme une zone de tutoriel où le joueur peut se familiariser avec les différentes mécaniques du jeu. Voici les éléments clés de cette première partie :
\\

 -Choix de la Classe de Personnage : Le joueur peut choisir parmi 9 classes différentes, chacune avec ses propres compétences et styles de jeu. Les classes comprennent des guerriers, des mages, des archers, et d'autres archétypes classiques, offrant une variété de choix stratégiques. 
  \\

  \textbf{Apprentissage des Mécaniques du Jeu : }
  \\

  - Maniement des Armes : Le joueur apprend à manier différentes armes comme les haches, épées, et arcs. Chaque arme a ses propres avantages et inconvénients, et le choix de l'arme dépend de la classe choisie. 
\\

  - Équipements par Classe : Le joueur découvre les différents types d'armures (lourdes, légères) adaptées à chaque classe, influençant la mobilité et la protection. 
\\

  - Éléments : Le jeu introduit les éléments (feu, eau, vent), chacun ayant des effets spécifiques en combat et sur l'environnement. 
\\

  - Consommables : Le joueur apprend à utiliser des potions de régénération, de défense, et d'autres consommables pour survivre et se renforcer. 
\\

  - Commerce en Ville : Interactions avec les forgerons, marchands, et autres commerçants pour acheter et améliorer l'équipement. 
\\

  -  Mouvements du Personnage : Le joueur apprend à se déplacer, sauter, courir, et interagir avec l'environnement. 
  \\

-Combat Tutoriel : 

  - À la fin du tutoriel, le joueur participe à un combat d'entraînement où il met en pratique la gestion des mouvements, l'utilisation des compétences, des pouvoirs élémentaires, des armes, et des consommables. Ce combat sert de test final pour les compétences acquises. 
-Lore et Contexte :

  - En plus des mécanismes de jeu, cette première partie introduit les bases du lore de Lands of Azerith, plaçant le joueur dans un monde riche en histoire et en mystères à découvrir. 



Partie 2 : Champignome Forest 
\\
 

Après le tutoriel, le joueur est téléporté à Champignome Forest à la suite d'une défaite obligatoire à Emberwood. Cette défaite fait partie intégrante du scénario et prépare le joueur à l'intrigue principale. 

  

- Exploration de Champignome Forest : 

 La forêt est un biome mystérieux et magique, rempli de champignons géants et de créatures étranges. Le joueur doit naviguer à travers cette forêt, en utilisant les compétences apprises à Emberwood. 
\\
 
  

- Rencontre avec les Smurfcats : 

- Les Smurfcats sont des créatures amicales et importantes pour l'histoire de Champignome Forest. Ces petits chats bleus marchant sur deux pattes avec un champignon en guise de chapeau aident le joueur et introduisent la mécanique des quêtes secondaires. 

  - Quêtes Secondaires : Le joueur peut choisir d'aider les Smurfcats, ce qui offre des récompenses sous forme d'équipement bonus et influence la fin du jeu. Ces quêtes secondaires ajoutent de la profondeur à l'histoire et permettent au joueur d'explorer davantage le lore et les mystères de la forêt. 
  \\
 
 

- Objectif Principal : 

 Le joueur doit se diriger vers Gravity Crater, en passant par l'Ancient City pour récupérer un brouillard quantique calibré. Ce brouillard est une formation paranormale cruciale dans le lore du jeu et est la clé pour retourner à Emberwood. 
 \\
 

 

Suite de l'Acte 1 
\\
 

  

De retour à Emberwood, le joueur se voit refuser l'entrée et doit se rendre à Ruined Runes. Cette partie de l'acte 1 introduit de nouveaux défis et prépare le joueur pour le premier grand donjon du jeu. 
\\
 

  

- Ruined Runes et Rune Cave : 
\\
 

  - Le joueur explore Ruined Runes, une zone pleine de dangers et de mystères, avant de pénétrer dans Rune Cave, le premier donjon majeur. 
  \\
 

  - Premier Boss : A.F.I.T. (Algorithme Fabricateur à Intervalle Tertiaire) :** A.F.I.T. est une machine ancienne maîtrisant l'élément mécanique, et le joueur doit utiliser toutes les compétences et stratégies apprises pour le vaincre. Ce combat de boss enseigne les mécaniques des combats de boss, déjà introduites partiellement lors du tutoriel. 
  \\
 

  - Victoire et Récompense : Après avoir vaincu A.F.I.T., le joueur obtient un parchemin crucial pour la suite de l'histoire, qui déverrouille de nouvelles quêtes et zones. 
  \\
 

  

- Traversée de la Black Forest et du White Volcano : 
\\
 

  - Le joueur traverse la Black Forest, une zone sombre et dangereuse, puis le White Volcano, une région enneigée avec ses propres défis. Ces zones contiennent des quêtes secondaires qui permettent de récupérer plus d'équipement et d'informations sur le lore. 
  \\
 

  - Quêtes Secondaires et Lore : En explorant ces zones, le joueur découvre des notes et dialogues enrichissant l'histoire et le contexte du jeu. 
  \\
 
 

- Arrivée au King's Palace : 

 L'acte 1 se termine lorsque le joueur atteint le King's Palace, prêt à lancer l'acte 2. Ce lieu marque une étape importante dans l'histoire, offrant une transition vers de nouveaux défis et explorations. 
 \\
 

  

Progression et Lore 
\\
 

  

- Système de Quêtes : 
\\
 

  - La progression dans Lands of Azerith se fait via un système de quêtes principales et secondaires. Les quêtes principales font avancer l'histoire, tandis que les quêtes secondaires offrent des opportunités d'exploration et de personnalisation supplémentaires. 
  \\
 

  - Récompenses de Quêtes : Chaque quête accomplie offre des récompenses telles que des bonus d'expérience, de l'équipement, des consommables ou des objets spéciaux. Ces récompenses sont souvent nécessaires pour progresser dans d'autres quêtes. 
  \\
 

  

- Découverte du Lore : 
\\
 

  Le joueur en apprend davantage sur le lore du jeu en accomplissant des quêtes secondaires, en discutant avec des personnages et en collectant des notes disséminées sur la carte. Ces notes sont automatiquement stockées dans un carnet accessible à tout moment, permettant au joueur de suivre les détails de l'histoire. 
  \\
 

  

Résumé des Zones et Biomes 
\\
 

  

Lands of Azerith comprend 21 zones distinctes, chacune jouant un rôle crucial dans la progression de l'histoire et les quêtes. Voici un résumé des zones clés de l'acte 1 : 
\\
 
  

- Emberwood : Départ et tutoriel, introduction des mécanismes de jeu et du lore. 
\\
 

- Champignome Forest : Exploration et introduction aux quêtes secondaires avec les Smurfcats. 
\\
 

- Ruined Runes et Rune Cave : Premier donjon et boss, enseignement des mécaniques de combat de boss. 
\\
 

- Black Forest et White Volcano : Traversée pour atteindre le King's Palace, avec des quêtes secondaires et une exploration du lore. 
\\
 

 


\subsection{Changements depuis la dernière soutenance}

Depuis la dernière soutenance, plusieurs changements significatifs ont eu lieu au sein de notre équipe et dans le développement du projet. 
Tout d'abord, nous avons dû faire face au départ d'Alexandre Colsch, qui a quitté EPITA  pour des raisons personnelles. 
Ce départ a été compensé par l'arrivée de Louise Fussien, dont les compétences en programmation et design ont été précieuses pour la progression du projet.
\\

Ce changement fait suite à une autre transition importante : Mohamed Aziz ben Amor, un membre clé de l'équipe, était parti juste avant notre première soutenance. 
La dynamique de notre équipe a donc été profondément impactée, nécessitant une adaptation rapide et une redistribution des responsabilités.
\\

Malgré ces bouleversements, nous avons réussi à maintenir un rythme de travail soutenu et à progresser de manière significative.
\\

\subsection{Annonce de plan}

Dans ce cahier des charges, nous allons vous présenter un compte rendu détaillé de l'avancement de notre projet, en décrivant les différentes étapes de son développement, les obstacles rencontrés et les solutions mises en place. Nous aborderons également les aspects organisationnels et techniques qui ont structuré notre travail.
\newline

Tout d'abord, nous discuterons des deadlines et de l'organisation. Nous commencerons par une analyse des objectifs initiaux et des progrès réalisés jusqu'à présent. Ensuite, nous ferons un point sur l'état actuel du projet. Nous expliquerons également les délais que nous avons rencontrés et les raisons pour lesquelles ils ont été nécessaires.
\\

Ensuite, nous examinerons l'organisation du groupe. Nous décrirons les méthodes de communication utilisées au sein de l'équipe pour garantir une collaboration efficace. Nous présenterons également la répartition des tâches entre les différents membres de l'équipe, et comment nous avons utilisé GitHub pour la gestion du code source et le versionnage.
\\

Nous aborderons ensuite le nouveau cahier technique, où nous fournirons une description détaillée des aspects techniques du projet. Cela inclura des sections spécifiques sur les différents modules et composants que nous avons développés.
\\

Pour le design, nous partagerons nos sources d'inspiration et expliquerons le processus de création et d'implémentation de l'interface et de l'expérience utilisateur. Nous traiterons également de l'intégration du son et d'autres éléments multimédias.
\\

Enfin, nous discuterons du site web du projet. Nous expliquerons l'objectif du site web, son développement et son apparence finale.
\\

La conclusion fournira un récapitulatif de tout ce que nous avons accompli, ainsi que des remerciements aux personnes et aux ressources qui nous ont soutenus tout au long de ce projet.
\\

Nous inclurons également des annexes pour fournir des informations supplémentaires et les documents pertinents pour une compréhension complète du projet.
\\

Ce document vise à fournir une vue d'ensemble complète et détaillée de notre travail, en mettant en lumière notre méthodologie, nos défis, et les solutions que nous avons trouvées pour les surmonter.
