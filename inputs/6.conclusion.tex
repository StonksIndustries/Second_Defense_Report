% Written by Louise Fussien

% This file CAN NOT be compiled on its own
% It is included by ../Book_of_Specifications.tex

\subsection{Récapitulatif}

Au terme de notre année académique à l'EPITA, nous sommes fiers de présenter les résultats de notre projet \gameName.
Ce projet a été une aventure marquée par des défis techniques, des adaptations organisationnelles, et des innovations créatives.
Nous avons entrepris de développer un jeu vidéo immersif qui non seulement nous a permis de mettre en pratique nos compétences en programmation et en design de jeux,
mais qui a aussi été un véritable terrain d'apprentissage en gestion de projet et en collaboration d'équipe.
\\

Notre projet a commencé avec une vision claire : créer un jeu d'aventure où les joueurs peuvent explorer un monde riche et interactif,
incarner des héros diversifiés, et participer à des quêtes palpitantes. L'acte 1 du jeu, centré sur le village d'Emberwood et ses environs,
a été le cœur de notre développement. Nous avons intégré des mécanismes variés tels que le choix de classes de personnages, la gestion des équipements,
l'utilisation des éléments, et un système de commerce en ville. Chaque étape du jeu a été soigneusement conçue pour offrir une expérience de jeu captivante et fluide.
\\

Le choix de Godot Engine comme moteur de jeu a été déterminant. Godot nous a offert une flexibilité et une puissance adaptées à nos besoins,
sans les contraintes financières des moteurs propriétaires. Nous avons pu exploiter ses capacités pour créer des environnements détaillés et
des mécaniques de jeu complexes. La gestion de version via GitHub a été essentielle pour coordonner nos efforts, permettant une collaboration
efficace malgré les défis liés aux départs de membres et à l'intégration de nouveaux.
\\

La dynamique de notre équipe a connu plusieurs bouleversements.
Le départ d'Alexandre Colsch et de Mohamed Aziz ben Amor a été compensé par l'arrivée de Louise Fussien, qui a apporté une nouvelle énergie et des compétences cruciales. L'utilisation de Discord pour la communication instantanée et des réunions en personne à EPITA a renforcé notre cohésion et notre capacité à résoudre les problèmes rapidement.
La répartition des tâches a été ajustée pour tirer le meilleur parti des compétences de chaque membre, garantissant une progression constante malgré les obstacles.
\\

L'environnement de travail, dans le contexte du projet de développement de jeux vidéo, est caractérisé par l'utilisation du moteur de jeu Godot et de l'environnement de
développement intégré (IDE) Rider. Godot, un moteur de jeu open-source, est reconnu pour sa facilité d'utilisation et sa capacité à gérer des projets de jeux vidéo en 2D et
en 3D. L'adoption de Godot a été motivée par sa communauté active et ses ressources abondantes, qui facilitent l'apprentissage et la résolution de problèmes. En parallèle,
Rider, un IDE multiplateforme développé par JetBrains, a été choisi pour ses fonctionnalités étendues de support de langages de programmation, en particulier C\#.
L'intégration de Godot avec Rider a permis d'exploiter pleinement les capacités de scripting C\# de Godot, offrant ainsi une expérience de développement plus
enrichissante et productive.
\\

Cependant, le développement n'a pas été exempt de défis, notamment ceux liés aux changements de version de Godot. Comme le moteur est en constante évolution,
les mises à jour fréquentes, bien que bénéfiques en termes de nouvelles fonctionnalités et d'améliorations de performance, peuvent introduire des modifications
significatives qui impactent la compatibilité du code existant. Pour surmonter ces difficultés, l'équipe a mis en place plusieurs stratégies, telles que des sessions de
formation interne, la création d'un environnement de test pour évaluer l'impact des changements sur le projet, et l'encouragement à la révision de code et au pair programming.
La manipulation des fichiers JSON a également été un aspect essentiel du projet. JSON, un format de données léger, a été largement utilisé pour stocker et échanger
des données structurées. En dépit des défis initiaux liés à la familiarité avec ce format de fichier, l'équipe a mené une étude approfondie de JSON et de ses applications.
Le processus d'intégration de fichiers JSON dans le code a impliqué la lecture et l'écriture de fichiers JSON, la sélection des attributs à sauvegarder, et la récupération
des objets à partir de fichiers JSON.
\\

En ce qui concerne l'architecture du jeu, l'organisation claire et la gestion efficace des ressources ont été facilitées par une structure de dossiers bien définie.
Le jeu vidéo s'est appuyé sur divers dossiers, notamment Assets pour les ressources visuelles et sonores, Caractères pour les images des personnages,
Items pour les fichiers JSON associés aux différents objets, et Loottables pour les données de niveau de détail des monstres.
\\

Dès l'ouverture du menu, l'effet parallax enchante visuellement en animant différents plans à des vitesses distinctes,
créant ainsi une profondeur saisissante. Cette technique, intégrée de manière astucieuse grâce à Godot, plonge les joueurs dans une perspective 2D
vivante dès le premier instant.
\\

En parallèle, notre approche du multijoueur repose sur une architecture robuste basée sur l'envoi de paquets, simplifiant
la connexion et la synchronisation entre les joueurs sans compromettre la fluidité du gameplay. Cette solution permet à jusqu'à huit joueurs
de se connecter sans tracas, chacun pouvant rejoindre une partie en utilisant un système intuitif sans avoir à saisir manuellement des adresses IP.
\\

Les contrôles du jeu sont pensés pour une jouabilité intuitive : de simples touches comme Z, Q, S, D pour les déplacements, Maj pour la course,
et Tab pour l'inventaire assurent une navigation fluide dans les environnements variés du jeu. De plus, des actions spécifiques comme l'interaction
avec le clic gauche de la souris sont finement intégrées pour une expérience utilisateur enrichie.
\\

L'intelligence artificielle des monstres apporte une autre dimension au jeu, avec des comportements variés (agressif, neutre, passif, peureux) qui
réagissent dynamiquement aux actions des joueurs. Chaque type de monstre est méticuleusement conçu pour enrichir l'interaction, utilisant des mécanismes
de détection de collision et de gestion de signaux pour ajuster leur comportement en temps réel.
\\



L'intégration des mécanismes avancés comme la navigation A* pour le déplacement des monstres, la gestion détaillée de l'inventaire du joueur, et la
création dynamique de maps avec transitions fluides dans notre jeu sous Godot représente un effort significatif pour offrir une expérience de jeu
immersive et engageante. Chaque élément technique, qu'il s'agisse de l'algorithme de pathfinding A* pour des déplacements intelligents,
de la gestion des objets collectés via un inventaire structuré, ou encore de l'animation fluide des
personnages et des environnements, contribue à créer un monde virtuel cohérent et captivant pour les joueurs.
\\

L'utilisation de tilesets et de textures bien intégrées garantit non seulement une esthétique visuelle agréable
mais aussi une navigation fluide entre les différents environnements du jeu. Les transitions entre les maps sont gérées
avec précision, offrant une immersion sans heurts tandis que les personnages et les monstres animés ajoutent une dimension réaliste et interactive à l'expérience de jeu.
\\

En combinant ces aspects techniques avec une mécanique de jeu bien pensée incluant des checkpoints pour la progression du joueur et une mécanique de loot gratifiante
à partir des monstres vaincus, notre jeu vise à captiver les joueurs tout en leur offrant un défi stratégique et une exploration enrichissante. En définitive, cette
approche technique avancée sous Godot non seulement améliore la jouabilité mais aussi renforce l'immersion et le plaisir des joueurs, faisant de chaque session de jeu une
expérience mémorable et divertissante.
L'aspect visuel et sonore de \gameName a été élaboré avec soin pour immerger les joueurs dans notre univers.
\\

Nous avons puisé notre inspiration dans diverses sources pour créer un design original et attrayant.
L'intégration du son, des effets visuels, et une interface utilisateur intuitive ont été des éléments clés pour enrichir l'expérience de jeu.
Le site web du projet a également été développé pour offrir une vitrine de notre travail, facilitant l'accès à des informations sur le jeu et son développement.
\\

Les défis que nous avons rencontrés, notamment les délais imprévus et les changements d'équipe, ont été surmontés grâce à une planification rigoureuse
et une flexibilité dans notre approche. Nous avons appris à nous adapter rapidement, à redistribuer les responsabilités, et à maintenir notre engagement
envers la qualité du projet. Chaque itération de notre cahier technique a été une occasion de réévaluer et d'améliorer notre travail, assurant que nous
restions alignés sur nos objectifs initiaux tout en intégrant de nouvelles idées et solutions.
\\

\subsection{Les joies et les peines}

Le chemin parcouru pour mener à bien notre projet a été parsemé de défis et de moments gratifiants, marquant une expérience enrichissante où la gestion du
temps et des ressources humaines a été cruciale. Voici un récit des hauts et des bas que nous avons traversés ensemble.
\\

Défis Rencontrés :

Gérer à la fois les études et le projet simultanément a été une tâche ardue. Trouver des plages horaires communes pour travailler efficacement a souvent
représenté un défi majeur, surtout avec les contraintes personnelles et académiques de chacun. Le départ de deux membres du groupe, Momo et Alex,
a également constitué un coup dur, nécessitant un réajustement rapide pour maintenir le cap du projet.
\\

Adaptation et Résilience

L'arrivée de Louise dans notre équipe a été une bouffée d'air frais. Sa contribution a apporté de nouvelles idées et une énergie positive, renforçant
notre capacité à relever les défis restants. Ce changement a été une occasion précieuse pour apprendre à gérer efficacement les transitions d'équipe
tout en maintenant la cohésion et la productivité.
\\

Apprentissage de la Gestion du Temps et du Groupe

Naviguer à travers ces dynamiques de groupe nous a offert une expérience précieuse dans la gestion du temps et des ressources humaines. Apprendre à
coordonner les efforts, à déléguer les tâches et à maintenir une communication claire a été essentiel pour atteindre nos objectifs collectifs. Cette
expérience nous a préparés de manière significative à l'environnement dynamique et collaboratif de l'entreprise.
\\

Satisfaction dans la Conformité aux Attentes

Finalement, livrer un projet qui répond aux attentes fixées représente une grande satisfaction. Cela démontre notre capacité à transformer des défis
en opportunités, à faire preuve de résilience face aux obstacles et à atteindre des résultats de qualité. Cette réalisation renforce notre confiance et
notre préparation pour les défis futurs, tant sur le plan académique que professionnel.


\subsection{Remerciements}



À ceux qui, par leur soutien et leur guidance, ont éclairé notre chemin tout au long de ce projet ambitieux, nous adressons nos remerciements sincères
et empreints de reconnaissance. À nos mentors et encadrants de l'EPITA, vous avez été bien plus que des guides : des conseillers précieux dont la sagacité
a éclairé nos démarches et nourri nos réflexions. Vos encouragements et vos conseils ont été les pivots sur lesquels s'est articulée notre entreprise, et sans votre
expertise, notre parcours aurait été semé d'embûches insurmontables.
\\

À nos proches, familles et amis, nous exprimons notre profonde gratitude pour votre soutien indéfectible. Votre patience et votre compréhension durant
les périodes intenses de développement ont été notre bouclier contre les vents contraires. Chaque succès que nous célébrons aujourd'hui porte l'empreinte
de votre affection et de votre encouragement constant.
\\

Nous ne saurions également passer sous silence l'importance cruciale des ressources open-source et des communautés de développeurs qui ont enrichi notre expérience.
Par leur générosité et leur partage de connaissances, ces précieux collaborateurs ont illuminé notre chemin, offrant des solutions éprouvées et des perspectives
nouvelles qui ont enrichi notre approche du développement.
\\

Les forums de discussion et les plateformes de partage de code ont été des carrefours d'échange essentiels, où chaque question technique trouvait réponse et chaque
idée trouvait écho. La richesse de ces interactions a été pour nous une source constante d'inspiration et de résolution de problèmes, permettant à notre projet de
croître et de s'épanouir au-delà de nos attentes initiales.
\\

Enfin, nous tenons à exprimer notre profonde reconnaissance envers les développeurs dédiés qui ont créé et maintenu les outils open-source que nous avons utilisés.
Leur dévouement et leur expertise ont été des piliers essentiels de la fiabilité et de la qualité de notre travail. Leur contribution silencieuse mais indispensable
a façonné notre succès et notre satisfaction aujourd'hui.
\\

En conclusion, \gameName incarne bien plus qu'un simple projet académique. Il symbolise notre engagement commun envers l'innovation et la création, ainsi
que notre capacité à surmonter les défis avec ingéniosité et persévérance. Que notre gratitude résonne comme une reconnaissance profonde envers tous ceux qui ont
contribué, directement ou indirectement, à la réalisation de cette œuvre collective.
\\

Avec respect et reconnaissance,
\\

Ayemane Bouarbi, Louise Fussien, Michaël Museux, Martin Pasquier