\subsection{Récapitulatif}
Au terme de notre année académique à l'EPITA, nous sommes fiers de présenter les résultats de notre projet "Lands of Azerith". 
Ce projet a été une aventure marquée par des défis techniques, des adaptations organisationnelles, et des innovations créatives. 
Nous avons entrepris de développer un jeu vidéo immersif qui non seulement nous a permis de mettre en pratique nos compétences en programmation et en design de jeux, 
mais qui a aussi été un véritable terrain d'apprentissage en gestion de projet et en collaboration d'équipe. 
\\

Notre projet a commencé avec une vision claire : créer un jeu d'aventure où les joueurs peuvent explorer un monde riche et interactif, 
incarner des héros diversifiés, et participer à des quêtes palpitantes. L'acte 1 du jeu, centré sur le village d'Emberwood et ses environs, 
a été le cœur de notre développement. Nous avons intégré des mécanismes variés tels que le choix de classes de personnages, la gestion des équipements, 
l'utilisation des éléments, et un système de commerce en ville. Chaque étape du jeu a été soigneusement conçue pour offrir une expérience de jeu captivante et fluide. 
\\

Le choix de Godot Engine comme moteur de jeu a été déterminant. Godot nous a offert une flexibilité et une puissance adaptées à nos besoins, 
sans les contraintes financières des moteurs propriétaires. Nous avons pu exploiter ses capacités pour créer des environnements détaillés et
 des mécaniques de jeu complexes. La gestion de version via GitHub a été essentielle pour coordonner nos efforts, permettant une collaboration 
 efficace malgré les défis liés aux départs de membres et à l'intégration de nouveaux. 
\\

La dynamique de notre équipe a connu plusieurs bouleversements.
Le départ d'Alexandre Cohen et de Mohamed Aziz ben Amor a été compensé par l'arrivée de Louise Fussien,
 qui a apporté une nouvelle énergie et des compétences cruciales. L'utilisation de Discord pour la communication instantanée 
 et des réunions en personne à EPITA a renforcé notre cohésion et notre capacité à résoudre les problèmes rapidement. 
 La répartition des tâches a été ajustée pour tirer le meilleur parti des compétences de chaque membre, garantissant une progression constante malgré les obstacles. 
\\

 

Partie technique recap 


 

  

L'aspect visuel et sonore de "Lands of Azerith" a été élaboré avec soin pour immerger les joueurs dans notre univers.
 Nous avons puisé notre inspiration dans diverses sources pour créer un design original et attrayant. 
 L'intégration du son, des effets visuels, et une interface utilisateur intuitive ont été des éléments clés pour enrichir l'expérience de jeu. 
 Le site web du projet a également été développé pour offrir une vitrine de notre travail, facilitant l'accès à des informations sur le jeu et son développement. 
\\

Les défis que nous avons rencontrés, notamment les délais imprévus et les changements d'équipe, ont été surmontés grâce à une planification rigoureuse 
et une flexibilité dans notre approche. Nous avons appris à nous adapter rapidement, à redistribuer les responsabilités, et à maintenir notre engagement
 envers la qualité du projet. Chaque itération de notre cahier technique a été une occasion de réévaluer et d'améliorer notre travail, assurant que nous 
 restions alignés sur nos objectifs initiaux tout en intégrant de nouvelles idées et solutions. 
\\

  

 
\subsection{Remerciements}

Nous tenons à exprimer notre gratitude envers tous ceux qui nous ont soutenus tout au long de ce projet. Nos professeurs et encadrants à l'EPITA ont été des sources inestimables de conseils et de motivation. Nous remercions également nos familles et amis pour leur soutien moral et leur patience durant les périodes intenses de développement. Enfin, nous reconnaissons l'importance des ressources open-source et des communautés de développeurs qui ont contribué indirectement à la réussite de notre projet.
\\

Le développement de "Lands of Azerith" ne s'arrête pas à l'acte 1. Nous envisageons de poursuivre notre travail en développant les actes suivants, en enrichissant l'histoire et les mécaniques de jeu. Les retours des utilisateurs seront essentiels pour guider nos améliorations et expansions futures. Notre expérience de cette année nous a équipé des compétences et de la confiance nécessaires pour aborder ces défis avec enthousiasme et détermination.
\\

En conclusion, "Lands of Azerith" est bien plus qu'un simple projet académique. C'est une réalisation collective qui témoigne de notre passion pour le développement de jeux vidéo, notre capacité à collaborer efficacement, et notre résilience face aux défis. Nous espérons que ce jeu apportera autant de plaisir aux joueurs qu'il nous en a apporté lors de sa création. Merci à tous ceux qui ont rendu cette aventure possible.

  

. 