Lands of Azerith est un jeu vidéo dont l'univers et le scénario sont au cœur de l'expérience de jeu. Dès le début du projet, une attention particulière a été portée à l'élaboration du scénario, ce qui a permis de le terminer relativement tôt. Le jeu est structuré en trois grands actes, chacun divisé en 5 à 7 parties, mêlant un scénario semi-linéaire à un vaste monde ouvert. L'action se déroule sur la terre d'Azerith, un environnement riche et varié composé de 21 zones distinctes, chacune représentant un biome différent.
La progression du joueur à travers l'histoire principale s'effectue via un système de quêtes principales, complétées par des quêtes secondaires qui offrent des équipements supplémentaires et peuvent même influencer la fin du jeu. Le premier acte, le plus abouti à ce stade du projet, débute dans le village d'Emberwood, où le joueur commence son aventure. Cette première partie fait office de tutoriel, permettant au joueur de choisir parmi 9 classes de personnage et d'apprendre les différentes mécaniques de jeu à travers des interactions avec les personnages non-joueurs.
Le tutoriel se termine par un combat où le joueur apprend à gérer les mouvements en combat, à utiliser les compétences, les pouvoirs élémentaires, les armes et les consommables. Ce premier segment pose également les bases du lore de Lands of Azerith.
La deuxième partie transporte le joueur dans la Champignome Forest après une défaite scénaristique obligatoire à Emberwood. Dans cette nouvelle zone, le joueur met en pratique ce qu'il a appris, rencontre diverses créatures et se lie d'amitié avec les Smurfcats, des créatures centrales à la trame de la Champignome Forest. Ces interactions introduisent les quêtes secondaires, offrant des récompenses et influençant la fin du jeu.
La trame principale de cette deuxième partie conduit le joueur à se diriger vers le Gravity Crater en passant par l'Ancient City pour obtenir un "brouillard quantique calibré", un élément crucial du lore permettant de retourner à Emberwood. Les trois dernières parties de l'acte 1 voient le joueur revenir à Emberwood, être refusé d'entrée, puis se diriger vers les Ruined 
Runes où il affronte son premier boss, A.F.I.T., dans le donjon de la Rune Cave. La victoire sur A.F.I.T. donne accès à un parchemin déterminant la suite de l'histoire, conduisant le joueur à travers la Black Forest et le White Volcano jusqu'au King's Palace, où l'acte 2 commence.
L'acte 2 met en avant l'aspect monde ouvert du jeu, le roi d'Azerith demandant au joueur de récupérer des objets dispersés aux quatre coins de la carte. Cet acte offre le plus de liberté au joueur, bien que la progression reste guidée pour éviter toute confusion. Le joueur explore de nouvelles zones, affronte plusieurs boss, apprend de nouvelles mécaniques et améliore son équipement.
L'acte 3, le dernier et le plus linéaire, présente la difficulté la plus élevée du jeu. Le joueur doit faire preuve de maîtrise et de préparation pour espérer terminer le jeu. Deux fins différentes sont possibles, influencées par les quêtes secondaires, et bien que ces fins n'affectent pas significativement le post-game, qui est entièrement multijoueur, elles offrent des conclusions variées à l'histoire.
Au fil du jeu, les quêtes secondaires, les discussions avec des personnages et la collecte de notes éparpillées sur la carte enrichissent le lore. Ces notes sont automatiquement ajoutées à un carnet que le joueur porte toujours sur lui. À la fin de chaque quête, qu'elle soit principale ou secondaire, le joueur reçoit des récompenses sous forme de bonus d'expérience, d'équipement, de consommables ou d'objets spéciaux utilisés dans d'autres quêtes.
